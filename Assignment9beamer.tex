\documentclass{beamer}
\usetheme{CambridgeUS}
\newcommand{\myvec}[1]{\ensuremath{\begin{pmatrix}#1\end{pmatrix}}}
\usepackage{amssymb}
\usepackage{amsmath}
\usepackage{graphicx}
\providecommand{\brak}[1]{\ensuremath{\left(#1\right)}}
\title{Assignment9}
\author{NIMMALA AVINASH(CS21BTECH11039)}
\institute{IITH}
\date{\today}
\logo{\large}

\begin{document}
\begin{large}
\maketitle
\begin{frame}{Subsections}
\begin{itemize}
\item Question
\item Solution
\end{itemize}
\end{frame}
\begin{frame}
{\LARGE \textbf{Question:\\}}
Show that if X(t) is BL as in and $\Delta$ = 2$\pi$/$\sigma$,then\\
\begin{align}
X(t) = 4 \sin^{2}{\frac{\sigma t}{2}} \sum_{n=-\infty}^{\infty}[\frac{X(n\Delta)}{(\sigma t - 2n\pi)^{2}} + \frac{X'(n\Delta)}{\sigma(\sigma t - 2n\pi)}]
\end{align}
\end{frame}
\begin{frame}{Solution}
    From Papoulis sampling expansion ,if we have N functions such as $P_{1}$($ \omega $,t),$ \cdots $ $P_{N}$($ \omega $,t) such that//
\begin{align}
H_{1}(\omega)P_{1}(\omega,\tau)+\cdots +H_{N}(\omega)P_{N}(\omega,\tau) &= 1\\
H_{1}(\omega+c)P_{1}(\omega,\tau)+\cdots +H_{N}(\omega+c)P_{N}(\omega,\tau) &= e^{jc\tau}
\end{align}
\end{frame}
\begin{frame}
W.K.T \\
N = 2,$H_{1}(\omega) = 1,H_{2}(\omega) = j\omega,c = \sigma$
\begin{align}
H_{1}(\omega)P_{1}(\omega,\tau)+H_{2}(\omega)P_{2}(\omega,\tau) &= 1\\
H_{1}(\omega+\sigma)P_{1}(\omega,\tau)+H_{2}(\omega+\sigma)P_{2}(\omega,\tau)&= e^{j\sigma\tau}\\
\implies P_{1}(\omega ,\tau) + j\omega P_{2}(\omega + \sigma) = 1\\
P_{1}(\omega,\tau)+j(\omega+\sigma)P_{2}(\omega,\tau) = e^{j\sigma\tau}\\
\implies P_{1}(\omega,\tau) = 1 - \frac{\omega}{\sigma}(e^{j\sigma\tau}-1)\\
P_{2}(\omega,\tau)  \frac{1}{j\sigma}(e^{j\sigma\tau}-1)\\
\end{align}
\end{frame}
\begin{frame}{Solution}
\begin{align}
    p_{k}(\tau) = \frac{1}{\sigma}\int_{-\sigma}^{0} P_{k}(\omega,\tau)e^{j\omega\tau}d\omega  \cdots
    where k = 1 \\
    \implies p_{1}(\tau) = \frac{1}{\sigma}\int_{-\sigma}^{0}p_{1}(\omega,\tau)e^{j\omega\tau}d\omega\\
    \implies p_{1}(\tau) = 
    \frac{1}{\sigma}\int_{-\sigma}^{0}(1-\frac{\omega}{\sigma}(e^{j\sigma\tau}-1))e^{j\omega\tau}d\omega\\
    \implies p_{1}(\tau) =
    \frac{1}{\sigma}(\frac{e^{j\omega\tau}}{j\tau})_{-\sigma}^{0}-(\frac{e^{j\sigma\tau}-1}{\sigma^{2}})\int_{-\sigma}^{0}e^{j\omega\tau}\omega d\omega\\
    \end{align}
    after integrating we get\\
    \end{frame}
    \begin{frame}
    \begin{align}
    p_{1}(\tau) &= \frac{2-2\cos(\sigma\tau)}{\sigma^{2}\tau^{2}}\\
    p_{1}(\tau) &= \frac{4\sin^{2}(\sigma\tau /2)}{\sigma^{2}\tau^{2}}\\
    similarly,   p_{2} (\tau)&=  \frac{4\sin^{2}(\sigma\tau /2)}{\sigma^{2}\tau}\\
  x(t+\tau) &= \sum_{n = -\infty}^{\infty}[y_{1}(t+nNT)P_{1}(\tau - nNT)+\cdots\\&\cdots+y_{N}(t+nNT)p_{N}(\tau - nNT)]\\
  with t &= 0,N = 2,P_{1} and P_{2},\\\tau &= t -   nT ,T = 2\pi /\sigma ,\Delta = 2\pi /\sigma
  \end{align}
  \end{frame}
  \begin{frame}{Solution}
  \begin{align}
  x(\tau) &= \sum_{n = -\infty}^{\infty}[{x(n\Delta)}P_{1}(\tau)+{x'(n\Delta)}P_{2}(\tau)]\\
  W.K.T,
    p_{1}(\tau) &= \frac{4\sin^{2}(\sigma\tau /2)}{\sigma^{2}\tau^{2}}\\
     p_{2} (\tau)&=  \frac{4\sin^{2}(\sigma\tau /2)}{\sigma^{2}\tau}\\
  x(\tau) &= \sum_{n=-\infty}^{\infty}[x(n\Delta)\frac{4\sin^{2}(\sigma\tau /2)}{\sigma^{2}\tau^{2}}+x'(n\Delta)\frac{4\sin^{2}(\sigma\tau /2)}{\sigma^{2}\tau}] \\
  \tau &= t - nT = t - 2n\pi/\sigma \implies \tau\sigma = \sigma t - 2n\pi\\
\end{align}
\end{frame}
\begin{frame}{Solution:}
    \begin{align}
    x(\tau) = 4\sin^{2}(\frac{\sigma\tau}{2})\sum_{n = -\infty}^{\infty}[\frac{x(n\Delta)}{(\sigma t-2n\pi)^{2}}+\frac{x'(n\Delta)}{\sigma(\sigma t - 2n\pi)}]
\end{align}
$$Hence Proved$$
\end{frame}
\end{large}
\end{document}






